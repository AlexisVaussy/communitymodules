\documentclass[a4paper,12pt]{scrartcl}

\usepackage[T1]{fontenc}        % Use EC fonts
\usepackage[latin1]{inputenc}  % Compatibility with Umlauts

\begin{document}


\section*{METKDemonstrator - Short Documentation}

Der METKDemonstrator dient der schnellen Pr�sentation einiger grundlegender Techniken des MedicalExplorationToolkits (METK).

\subsection*{Laden eine Falles}

Neben dem Demo-Datensatz (einem HNO-Fall) k�nnen beliebige eigene F�lle geladen werden, so sie denn mit dem NeckSegmenter bzw. NeckVision der Uni Magdeburg oder HepaVision von MeVis erstellt wurden. Beim Laden von HepaVision F�llen ist zu beachten, dass die XML-Datei des Falles schon im ObjMgr-Format vorliegen muss, da die Konvertierung des Ursprungs HepaVision-Formats nicht (mehr) funktioniert.

Die XML-Datei des Falles muss au�erdem im Verzeichnis der Segmentierungsmasken und Bilddaten liegen oder die Verweise auf Daten in Unterverzeichnissen m�ssen in der XML-Datei entsprechend vorhanden sein. Automatisch aus einem \texttt{datasets}-Pfad geladen, wie es bei MeVis gemacht wird, wird nichts. Die Segmentierungsergebnisse und alles was sonst noch gebraucht wird, wird relativ zur XML-Datei geladen.

Um einen HepaVision-Fall also laden zu k�nnen, muss also die ObjMgr-XML-Datei des Falles zu den anderen Daten kopiert und von dort geladen werden.

Das erstmalige Laden eines Falles dauert etwas l�nger, da hier die Inventor-Objekte als Dateien im Datenverzeichnis erzeugt werden. Danach geht das Laden und anzeigen der Strukturen um so schneller.

\subsection*{3D-Viewer}
Im 3D-Viewer kann wie gewohnt navigiert werden. Werden Objekte selektiert, so "`fliegt"' die Kamera an sie heran. Dazu muss nicht in den Selektionsmodus geschaltet werden. Einfaches klicken mit der "`Hand"' reicht.

\subsection*{Strukturbrowser}

Rechts oben befinden sich 2 Arten von Strukturbrowsern, wie sie im METK zum Einsatz kommen. Der Hierarchy-Browser bildet die Parent-Child-Struktur eines Datensatzes ab. Gew�hnlich werden hier die ROIs und die aus ihnen segmentierten Strukturen dargestellt. Im StruktureGroup-Browser werden nur die segmentierten Strukturen dargestellt. Sie werden gruppiert entsprechend ihres \texttt{StructrueGroup}-Eintrags im \texttt{Description}-Layer.

Durch das H�kchen k�nnen einzelne Strukturen oder ganze Gruppen sichtbar und unsichtbar gemacht werden.


\subsection*{Visualisierungseigenschaften}

In der Mitte rechts k�nnen die Visualisierungseigenschaften wie Farbe und Transparenz einzelner Strukturen ver�ndert werden. Es wird dabei die Struktur angepasst, die im aktuellen Browser selektiert ist. Beim StructureGroup-Browser muss (derzeit noch) wirklich eine Struktur (Blatt des Baums) ausgew�hlt sein. Beim Hierarchy-Browser k�nnen auch Elternknoten gew�hlt werden. Die eingestellten Eigenschaften werden dann f�r alle Childs gesetzt.


\subsection*{Selected Object}

In diesem kleinen Viewer wird das aktuell selektierte Objekt angezeigt, so es denn im 3D-Viewer selektiert wurde. Man k�nnte diesen Viewer auch zus�tzlich mit den Browsern koppeln, nur das w�re zu verwirrend f�r einen ersten Eindruck. Deswegen hier also nur das im 3D-Viewer selektierte Objekt.


\subsection*{Collections}

Der aktuelle Status aller Visualisierungsparameter einschlie�lich der Kamera kann durch eine Collection festgehalten und gespeichert werden. Wird eine schon bestehende Collection in der Liste selektiert, so wird sie geladen und animiert eingeblendet.


\end{document}
